\subsection{Succinctness}

We now prove that our construction produces succinct proofs.

Observe that full succinctness in the standard honest majority model is
impossible to prove for our construction. To see why, recall that an adversary
with sufficiently large mining power can significantly harm superquality as
described in Subsection~\ref{subsec.superquality-attack}. By causing this harm
in superquality of a sufficiently low level $\mu$, for example $\mu = 3$, the
adversary can cause the honest prover to digress in their proofs from high-level
superchains down to low-level superchains, causing a linear proof to be
produced. For instance, if superquality is harmed at $\mu = 3$, the prover will
need to digress down to level $\mu = 2$ and include the whole $2$-superchain,
which, in expectation, will be of size $|\chain|/2$.

Having established security in the general case of the standard honest majority
model, we therefore concentrate our succinctness claims to the particular case
where there is no adversary. This case, called the ``optimistic succinctness''
case following the lead of \cite{KLS}, models the system as follows. First, the
adversary has no mining power. In other words, we set $t = 0$. This establishes
that all honestly adopted chains contain only honestly generated blocks. Second,
the adversary is unable to perform adversarial network scheduling. Instead,
network scheduling is random. What this means is that messages diffused by
honest parties are reordered in a random permutation before they are delivered
to other honest parties, but the adversary has no saying as to this order. And
last, the adversary is unable to produce any proofs; that is, all proofs
consumed by the verifier are honestly-generated. We will now explore why proofs
produced in these circumstances are succinct.

\begin{restatable}[Number of levels]{theorem}{restateThmFewLevels}
    \label{thm.few-levels}
    The number of superblock levels which have at least $m$ blocks are at most
    $\log(|S|)$, where $S$ is the set of all blocks produced, with overwhelming
    probability in $m$.
\end{restatable}

\ifonecolumn
\import{./}{proofs/fewlevels.tex}
\fi

The above theorem establishes that the number of superchains is small. What
remains to be shown is that few blocks will be included at each superchain
level.

\begin{restatable}[Large upchain expansion]{theorem}{restateThmLargeExpansion}
    \label{thm.large-expansion}
    Let $\chain$ be an honestly generated chain and let
    $\chain' = \chain\upchain^{\mu - 1}[i:i + \ell]$ with $\ell \geq 4m$.
    Then:

    \begin{equation*}
      |\chain'\upchain^\mu| \geq m
    \end{equation*}

    with overwhelming probability in $m$.
\end{restatable}

\ifonecolumn
\import{./}{proofs/largeexpansion.tex}
\fi

\begin{restatable}[Small downchain support]{corollary}{restateThmSmallSupport}
    \label{crly.small-support}
    Assume an honestly generated chain $\chain$ and let $\chain' = \chain\upchain^\mu[i:i + m]$. Then:

    \begin{equation}
      |\chain'\downchain\upchain^{\mu - 1}| \leq 4m
    \end{equation}

    with overwhelming probability in $m$.
\end{restatable}

\ifonecolumn
\import{./}{proofs/smallsupport.tex}
\fi

This last theorem establishes the fact that the support of blocks needed under
the $m$-sized chain suffix to move from one level to the one below is small.
Based on this, we can establish our theorem on succinctness:

\begin{restatable}[Optimistic succinctness]{theorem}{restateThmSuccinctness}
    \label{thm.succinctness}
    Non-interactive proofs-of-proof-of-work produced by honest provers on
    honestly generated chains with random scheduling are succinct with the
    number of blocks bounded by $4m \log(|\chain|)$, with overwhelming
    probability in $m$.
\end{restatable}

\ifonecolumn
\import{./}{proofs/succinctness.tex}
\else
Full proofs of these facts are provided in Appendix~\ref{sec.proofs}.
\fi

\begin{remark}[Denial of Service Attacks]
Examine the non-optimistic case where the adversary is allowed to provide
proofs, yet, as in the optimistic case, they have no mining power or adversarial
scheduling capabilities. It is then still possible for that adversary to produce
large incorrect proofs that use up  the resources of a verifier with logarithmic
time. Therefore, if such attacks are of concern, the verifier can relax the
security guarantees to achieve succinctness in this model as follows: Whenever a
large proof is provided by a prover, the verifier terminates early with output
$\bot$ regardless of which proof is better, indicating that they are unable to
reach a conclusion. It is important to note that a verifier knowing ahead of
time that a proof will be large cannot conclude that the prover providing it is
malicious, as an attack on superquality of the blockchain could be taking place,
requiring an honest prover to provide long proofs.
\end{remark}
