\section{Succinctness}
\label{sec:succinctness}

We analyse the scheme in
Algorithm~\ref{alg.nipopow-infix-prover}. We will illustrate why our
construction is succinct in the honest setting. For techniques to make the
construction succinct in broader adversarial settings (where an adversary can
produce proofs), consult the full version of this paper.

Having established security in the general case of the standard honest majority
model, we now concentrate our succinctness claims to the particular
``optimistic'' case where the adversary does not use their (minority)
computational power or network power.

\begin{definition}[Optimistic execution]
  We will call an execution \emph{optimistic} if the adversary has $q = 0$
  random oracle queries and the messages diffused by honest parties are
  delivered in random (and not adversarial) order.
\end{definition}

Last, we will not allow the adversary to produce any proofs; that is, all proofs
consumed by the verifier are honestly-generated.
Based on this, we can establish our theorem on succinctness:

\begin{restatable}[Optimistic succinctness]{theorem}{restateThmSuccinctness}
    \label{thm.succinctness}
    In an optimistic execution, Non-Interactive Proofs of Proof-of-Work produced
    by honest provers are succinct with the number of blocks bounded by $4m
    \log(|\chain|)$, with overwhelming probability in $m$.
\end{restatable}
\ifonecolumn
% \import{./}{proofs/succinctness.tex}
\fi

In the above theorem, note the linear dependency between the round $r$ during
which a proof is generated and the length $|\chain|$ of the chain of each honest
prover.
