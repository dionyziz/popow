%\subsection{Succinctness}
\label{sec.succinctness}

\noindent{\bf Succinctness.}
We now prove that our construction produces succinct proofs.
%
We first observe that full succinctness in the standard honest majority model is
impossible to prove for our construction. To see why, recall that an adversary
with sufficiently large mining power can significantly harm superquality as
described in %Subsection
Section~\ref{subsec.superquality-attack}. By reducing
 superquality for a sufficiently low level $\mu$, for example $\mu = 3$, the
adversary can cause the honest prover to digress in their proofs from high-level
superchains down to low-level superchains, causing a linear proof to be
produced.
%For instance, if superquality is harmed at $\mu = 3$, the prover will
% need to digress down to level $\mu = 2$ and include the whole $2$-superchain,
% which, in expectation, will be of size $|\chain|/2$.

Having established security in the general case of the standard honest majority
model, we now concentrate our succinctness claims to the particular
``optimistic'' case where the adversary does not use their (minority)
computational power or network power. Therefore, the superquality of the chain
must be the same as a fully honestly-generated chain generated with no network
adversary. Last, for now, we will not allow the adversary to produce any proofs;
that is, all proofs consumed by the verifier are honestly-generated. We will
lift this last assumption shortly.

\begin{restatable}[Number of levels]{theorem}{restateThmFewLevels}
    \label{thm.few-levels}
    The number of superblock levels which have at least $m$ blocks are at most
    $\log(|S|)$, where $S$ is the set of all blocks produced, with overwhelming
    probability in $m$.
\end{restatable}

\ifonecolumn
\import{./}{proofs/fewlevels.tex}
\fi

The above theorem establishes that the number of superchains is small. What
remains to be shown is that few blocks will be included at each superchain
level.

\begin{restatable}[Large upchain expansion]{theorem}{restateThmLargeExpansion}
    \label{thm.large-expansion}
    Let $\chain$ be an honestly generated chain and let
    $\chain' = \chain\upchain^{\mu - 1}[i:i + \ell]$ with $\ell \geq 4m$.
    Then $|\chain'\upchain^\mu| \geq m$
    with overwhelming probability in $m$.
\end{restatable}

\ifonecolumn
\import{./}{proofs/largeexpansion.tex}
\fi

\begin{restatable}[Small downchain support]{lemma}{restateThmSmallSupport}
    \label{lem.small-support}
    Assume an honestly generated chain $\chain$ and let $\chain' = \chain\upchain^\mu[i:i + m]$. Then
    $|\chain'\downchain\upchain^{\mu - 1}| \leq 4m$
    with overwhelming probability in $m$.
\end{restatable}

\ifonecolumn
\import{./}{proofs/smallsupport.tex}
\fi

This last theorem establishes the fact that the support of blocks needed under
the $m$-sized chain suffix to move from one level to the one below is small.
Based on this, we can establish our theorem on succinctness:

\begin{restatable}[Optimistic succinctness]{theorem}{restateThmSuccinctness}
    \label{thm.succinctness}
    Non-interactive proofs-of-proof-of-work produced by honest provers in the
    optimistic case are succinct with the number of blocks bounded by $4m
    \log(|\chain|)$, with overwhelming probability in $m$.
\end{restatable}

\ifonecolumn
\import{./}{proofs/succinctness.tex}
\else
\fi

\textbf{Certificates of badness.}
We proved security in the general case, but succinctness only in the optimistic
case. Examine now the optimistic case extended with the adversary's ability to
provide proofs. It is then still possible for that adversary to produce large
dummy (incorrect) proofs that use up the resources of a verifier with
logarithmic time. A verifier knowing ahead of time that a proof will be large
cannot conclude that the prover providing it is malicious, as an attack on
superquality of the blockchain could be taking place (and we want to maintain
security in general), requiring an honest prover to provide long proofs.

To avoid the above undesirable scenario, we offer a generalization of our above
construction. Our extended construction allows the verifier to stop processing
input early, in a streaming fashion, thereby only requiring logarithmic
communication complexity. To achieve this, observe that honest proofs need to
be large only if there is a violation of \textit{goodness}. However, goodness is
not harmed when the chain is not under attack by the adversarial computational
power or network. As such, we require the prover to produce a
\textit{certificate of badness} in case there is a violation of
\textit{goodness} in the blockchain. This certificate will always be logarithmic
in size and must be sent prior to the rest of the proof by the prover to the
verifier. Because the certificate will be logarithmic in size even in the case
of an adversarial attack on the chain, the honest verifier can stop processing
the certificate after a logarithmic time bound. If the certificate is claimed to
be longer, the honest verifier can reject early by deciding that the prover is
adversarial. Looking at the certificate, the honest verifier determines whether
there is a possibility for a lack of goodness in the underlying chain. If
there's no adversarial computational power in use, the certificate is impossible
to produce.

The certificates of badness are produced simply. First, the honest verifier
finds the maximum level $\text{max-}\mu$ at which there are at least $m$
$\text{max-}\mu$-superblocks and includes it in the certificate. Then, because
there is a violation of goodness there must exist two levels $\mu < \mu'$ such
that $2^\mu|\chain\upchain^\mu| > (1 + \delta)2^{\mu'}|\chain\upchain^{\mu'}|$
in some part $\chain$ of the honestly adopted chain. But $\mu' - \mu \leq
\text{max-}\mu$. Therefore, there must exist two adjacent levels $\mu_1 < \mu_2$
which break goodness but with error parameter $(1 +
\delta)^{1/{\text{max-}\mu}}$. In particular, it will hold that
$2^{\mu_1}|\chain\upchain^{\mu_1}| > (1 +
\delta)^{1/{\text{max-}\mu}}2^{\mu_2}|\chain\upchain^{\mu_2}|$. This condition
is direct for the prover to find and trivial for the verifier to check and
completes the construction.
% Note that it is possible that a certificate of
% badness is produceable where two adjacent levels have more than $(1 +
% \delta)^{1/{\text{max-}\mu}}$ error even if there is no harm to global goodness;
% however, these certificates cannot be produced when no adversarial power is in
% use.
