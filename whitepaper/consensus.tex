\section{Consensus layer support} \label{sec.consensus}

\subsection{The interlink pointers data structure}
\label{sec.interlink}

In order to construct our protocol, we rely on the same \emph{interlink data
structure} used by PoPoW~\cite{KLS}. This is an additional hash-based data
structure that is proposed to include in the header of each block. The
interlink data structure is a skip-list~\cite{skiplist} that makes it efficient
for a verifier to process a sparse subset of the blockchain, rather than only
consecutive blocks.

Observe that in a blockchain protocol execution it is expected half of the
blocks will be of level $1$, $1/4$ of the blocks will be of level $2$, $1/8$
will be of level $3$ and $1/2^\mu$ blocks will be of level $\mu$. In
expectation, the number of superblock levels of a chain $\chain$ will be
$\Theta(\log(\chain))$~\cite{KLS}. Figure~\ref{fig.hierarchy} illustrates the
blockchain superblocks starting from level $1$ and going up to level $4$ in case
these blocks are distributed exactly according to expectation. Here, each level
contains half the blocks of the level below.

In our protocol, the verifier must roughly scan along one level at a time. To
enable this, instead of just the previous block, the interlink vector also
points to the most recent preceding block of every level $\mu$. Genesis is of
infinite level and hence a pointer to it is included in every block at the first
available index within the interlink data structure. The number of pointers that
need to be included per block is in expectation $\log(|\chain|)$.

\begin{figure}
    \caption{The hierarchical blockchain.
    Higher levels have achieved a lower target (higher difficulty) during mining.}
    \centering
    \iftwocolumn
        \includegraphics[width=\columnwidth,keepaspectratio]{figures/hierarchical-ledger.png}
    \else
        \includegraphics[width=0.7\columnwidth,keepaspectratio]{figures/hierarchical-ledger.png}
    \fi
    \label{fig.hierarchy}
\end{figure}

%The \textit{interlink} data structure is proposed to be included in each block,
%replacing the existing pointer to the previous block with a
%

%\iftr
The algorithm
for this construction is shown in Algorithm~\ref{alg.nipopow-interlink} and is
borrowed from~\cite{KLS}.
The interlink data structure turns the blockchain into a
skiplist-like~\cite{skiplist} data structure.

The updateInterlink algorithm accepts a block $B'$, which already has an
interlink data structure defined on it. The function evaluates the
interlink data structure which needs to be included as part of the next block.
It copies the existing interlink data structure and
then modifies its entries from level $0$ to $\textsf{level}(B')$ to
point to the block $B'$.

\import{./}{algorithms/alg.nipopow-helper.tex}
\import{./}{algorithms/alg.nipopow-interlink.tex}

\subsection{Superchain quality}
In order to argue formally about the security properties of blockchains that are
equipped with the interlink data structure we will introduce a new concept of
{\em superchain quality}, which generalizes the chain quality property from the
backbone model~\cite{backbone}. Superchain quality is a new contribution in this
paper and is essential for identifying and overcoming the attack on PoPoW.

We first define a notion of ``goodness'' that bounds the deviation of
superblocks of a certain level from their expected mean. Using this we then
define superchain quality.

Intuitively, these definitions tell us that $\mu$-superblocks occur
approximately once every $2^\mu$ blocks. Below, we make this notion more formal.

\begin{definition}[Locally good superchain]
A superchain $\chain'$ of level
$\mu$ with underlying chain $\chain$ is said to be $\mu$-\textnormal{locally-good}
with respect to security parameter $\delta$, written
$\textsf{local-good}_{\delta}(\chain', \chain, \mu)$, if $|\chain'| > (1 -
\delta)2^{-\mu}|\chain|$.
\end{definition}

\begin{definition}[Superchain quality]
The $(\delta, m)$ superquality property $Q^\mu_{scq}$ of a chain $\chain$
pertaining to level $\mu$ with security parameters $\delta \in \mathbb{R}$ and
$m \in \mathbb{N}$ states that for all $m' \geq m$, it holds that
$\textsf{local-good}_{\delta}(C\upchain^\mu[-m':],
C\upchain^\mu[-m':]\downchain, \mu)$. That is, all sufficiently large suffixes
are locally good.
\end{definition}

While superchain quality guarantees that a superchain has a sufficient number
of blocks w.r.t. the level 0 chain, we will also need to be able to compare
it with other superchains. For this reason we introduce multilevel quality.

\begin{definition}[Multilevel quality]
A $\mu$-superchain $\chain'$ is said to have \textit{multilevel quality}, written
$\textnormal{multi-good}_{\delta, k_1}(\chain, \chain', \mu)$ with respect to an
underlying chain $\chain = \chain'\downchain$ with security parameters $k_1,
\delta$ if for all $\mu' < \mu$ it holds that for any $\chain^* \subseteq \chain$,
if $|\chain^*\upchain^{\mu'}| \geq k_1$, then $|\chain^*\upchain^{\mu}| \geq (1 -
\delta)2^{\mu - \mu'}|\chain^*\upchain^{\mu'}|$.
\end{definition}

% For $\delta < 0.5$, multilevel quality implies that
% $|\chain^*\upchain^{\mu}| \geq 1$.

Putting the above together we conclude with the notion of a {\em good}
superchain.

\begin{definition}[Good superchain]
\label{lem.good}
A $\mu$-superchain $\chain'$
is said to be \textit{good}, written $\textnormal{good}_{\delta, k_1}(\chain,
\chain', \mu)$, with respect to an underlying chain $\chain = \chain'\downchain$
if it has both superquality and multilevel quality with parameters $(\delta,
m)$.
\end{definition}

\begin{lemma}[Local goodness]
\label{lem.localgood}
Assume $\chain$ contains only honestly-generated blocks in an optimistic
execution. For all levels $\mu$, for all constant $\delta > 0$, all continuous
subchains $\chain' = \chain[i:j]$ with $|\chain'| \geq m$ are locally good,
$\textsf{local-good}_{\delta}(\chain', \chain, \mu)$, with overwhelming
probability in $m$.
\end{lemma}
\import{./}{proofs/localgood.tex}

\begin{lemma}[Multilevel quality]\label{lem.multilevel}
For all $\mu, 0 < \delta \leq 0.5$, chain $\chain$ containing only
honestly-generated blocks in an optimistic execution has $(\delta, k_1)$
multilevel quality at level $\mu$ with overwhelming probability in $k_1$.
\end{lemma}
\begin{proof}
Identical.
\Qed
\end{proof}

\begin{lemma}[Superquality]
\label{lem.superquality}
For all $\mu, \delta > 0$, a chain $\chain$ adopted in an optimistic execution
has $(\delta, m)$-superquality at level $\mu$ with overwhelming probability in
$m$.
\end{lemma}
\import{./}{proofs/superquality.tex}

\begin{lemma}[Optimistic superchain distribution]
\label{lem.superchain-distribution}
For a level $\mu$, and $0 < \delta < 0.5$, a chain
$\chain$ containing only honestly-generated blocks adopted by an honest party in
an execution with random scheduling is $(\delta, m)$-good at level
$\mu$ with overwhelming probability in $m$.
\end{lemma}
\begin{proof}
This is a direct consequence of Lemma~\ref{lem.superquality} and
Lemma~\ref{lem.multilevel}. \Qed
\end{proof}

It is not hard to see that the above good statistical properties are attained
with overwhelming probability by all chains that are generated in optimistic
environments, i.e. if no adversary tries to violate them.

\begin{lemma}[Optimistic superchain distribution]
\label{lem.superchain-distribution}
For a level $\mu$, and $0 < \delta < 0.5$, a chain
$\chain$ containing only honestly-generated blocks adopted by an honest party in
an execution with random scheduling is $(\delta, m)$-good at level
$\mu$ with overwhelming probability in $m$.
\end{lemma}
\begin{proof}
This is a direct consequence of Lemma~\ref{lem.superquality} and
Lemma~\ref{lem.multilevel}. \Qed
\end{proof}
