\section{Backbone integration}
\label{sec.appendix-backbone}

%We use this model to explore which predicates are reliable and can be proven
%succinctly.

In this section, we illustrate the detailed formalisms to integrate our proof
systems with the backbone model \cite{backbone}.

The backbone protocol of a miner node is shown in Algorithm~\ref{alg.backbone}.
The honest miner maintains the longest chain from the network and tries to mine
on top of it. In Algorithm~\ref{alg.backbone-prover}, we illustrate our new
entity for the model, the full node or \textit{prover}.  While the full node is
not mining, it maintains a state with the longest chain from the network.
Furthermore, whenever it is asked to prove a predicate $Q$ about its local
chain $\chain$, it calls a Prove function to provide the proof.  We leave this
Prove function undefined here, as it is part of the concrete protocol
construction. On the other hand, in Algorithm~\ref{alg.generic-verifier} we
illustrate another new entity to the model, the generic \textit{verifier},
which is stateless, but receives proofs from the provers on the network (via
the environment) and takes a decision about a predicate on the blockchain it
believes to be the longest.

\import{./}{algorithms/alg.backbone.tex}
\import{./}{algorithms/alg.backbone-prover.tex}
\import{./}{algorithms/alg.verifier-framework.tex}

The prover which additionally maintains the \textit{blockById},
\textit{depth} and \textit{realLink} data structures is illustrated in
Algorithm~\ref{alg.backbone-velvet-prover}.

\import{./}{algorithms/alg.backbone-velvet-prover.tex}
\section{Full proofs}
\label{sec.proofs}

\restateThmSecurity*

\begin{proof}
    Let $\overline{\Pi}_A, \overline{\Pi}_B$ be an adversarial and an honest
    proof respectively.

    Let $b$ be the LCA of $\overline{\Pi}_A$ and $\overline{\Pi}_B$ and assume
    that $A$ is the adversary. Let $\mu$ be the highest level at which one of $A$
    or $B$ is able to present $m$ blocks of level $\mu$ after $b$. We will show
    that if such a level exists, then the winning proof will necessarily be
    $\overline{\Pi}_B$.

    Indeed, because $\mu$ is the highest level at which any of the parties was
    able to present at least $m$ blocks of that level after $b$, this means
    that at every higher level $\mu + i$, it was true that $b$ was within the
    last $m$ blocks of that level. Therefore, $B$, due to the way the proofs
    are constructed, would have included all the blocks following $b$ of that
    level. Hence, all the blocks of the honest chain following $b$ of level
    $\mu$ have been included in the proof $\overline{\Pi}_B$.

    This means that the blocks presented after $b$ by $B$ at level $\mu$ must
    be disjoint from those presented after $b$ by $A$ at level $\mu$, since $b$
    is defined as the LCA. But because $B$ presented all the honest chain
    blocks of level $\mu$, therefore the $m$ blocks presented by $A$ at level
    $\mu$ must be adversarially mined.  However, because both the honest
    parties and the adversary started mining blocks of level $\mu$
    simultaneously after the round during which block $b$ was generated, the
    probability that the adversary is able to produce $m$ $\mu$-level blocks
    after block $b$ faster than the honest parties are able to produce $m$
    $\mu$-level blocks after block $b$ is negligible.

    Therefore, if there exists such a level $\mu$, then the honest party proof
    will necessarily win. If there is no such level $\mu$, then the comparison
    happens at the 0-level, and hence reduces to the security of a full
    verifier.
\end{proof}

\restateThmFewLevels*

\begin{proof}
    Let $S$ be the set of all blocks successfully produced by the honest
    parties or the adversary. Because each block id is generated by the random
    oracle, the probability that it is less than $T 2^{-\mu}$ is
    $2^{-\mu}$. These are independent Bernoulli trials. For each block
    $B \in S$, define $X^{\mu}_B \in \{0, 1\}$ to be the random variable
    indicating whether the block belongs to superblock level $\mu$ and let
    $D_\mu$ indicate their sum, which is a Binomial distribution with
    parameters $(|S|, 2^{-\mu})$ and expected value $E[D_{\mu}] =
    |S| 2^{-\mu}$.

    For level $\mu$ to exist in any valid proof, at least $m$ blocks of level
    $\mu$ must have been produced by the honest parties or the adversary. We
    will now show that $m$ blocks of level $\mu = \log(|S|)$ are produced with
    negligible probability in $m$.

    As all of the $X^{\mu}$ are independent, we can apply a Binomial Chernoff
    bound to the probability of the sum. Therefore we have

    $\Pr[D_\mu \geq (1 + \Delta)E[D_\mu]] \leq \exp(-\frac{\Delta^2}{2 +
    \Delta}E[D_\mu])$. But for this $\mu$ we have that $E[D_\mu] = 1$.
    Therefore $\Pr[D_\mu \geq 1 + \Delta] \leq \exp(-\frac{\Delta^2}{2 +
    \Delta})$. Requiring $1 + \Delta = m$, we get $\Pr[D_\mu \geq m] \leq
    \exp(-\frac{(m - 1)^2}{m + 1})$, which is negligible in $m$.
\end{proof}

\restateThmLargeExpansion*

\begin{proof}
    Assume the $(\mu - 1)$-level superchain had $4m$ blocks. Because each block
    of level $\mu - 1$ was generated as a query to the random oracle, it
    constitutes an independent Bernoulli trial and the number of blocks in
    level $\mu$, namely $\pi[\mu]$, is a Binomial distribution with parameters
    $(4m, 1/2)$. Applying a Chernoff bound to the
    distribution, we have that $\Pr[|\pi[\mu]| = m] \leq \Pr[|\pi[\mu]| \leq
    m]$. Observing that $E[\pi[mu]] = 2m$ the previous probability becomes
    $\Pr[|\pi[\mu]| \leq (1 - \frac{1}{2})2m] \leq \exp(-\frac{(1/2)^2}{2} 2m)$
    which is negligible in $m$.

    This probability bounds the probability of fewer
    than $m$ blocks occurring in the $\mu$ level restriction of $(\mu -
    1)$-level superchains of more than $4m$ blocks.
\end{proof}

\restateThmSmallSupport*

\begin{proof}
    Assume the $(\mu - 1)$-level superchain had at least $4m$ blocks. Then by
    Theorem~\ref{thm.large-expansion} it follows that more than $m$ blocks
    exist in level $\mu$ with overwhelming probability in $m$, which is a
    contradiction.
\end{proof}

\restateThmSuccinctness*

\begin{proof}
    Assume $\chain$ is the honest parties chain. From \ref{thm.few-levels}, the
    number of levels in the NIPoPoW is at most $\log(|\chain|)$ with
    overwhelming probability in $m$.

    First, observe that the count of blocks in the highest level will be less
    than $4m$ from Theorem~\ref{thm.large-expansion}; otherwise a higher
    superblock level would exist. Then for the $m$-long suffix of each
    superchain of level $\mu$, the supporting superchain of level $\mu - 1$
    will have at most $4m$ blocks from Corollary~\ref{crly.small-support}.

    Therefore the size of the proof is $4m \log(|\chain|)$, which is succinct.
\end{proof}
