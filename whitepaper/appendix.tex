% \appendix
% \section*{Appendix}

% Our Appendix is structured as follows.
% In Section~\ref{sec:infix} we discuss how to extend our scheme to prove facts
% buried deep within the blockchain. This ability can be easily added to our
% protocol, but is important in order to be able to show that a transaction took
% place in the past.
% In Section~\ref{sec:parameters} we measure the concrete probability of success
% of our scheme in order to provide concrete parameters for implementors.
% In Section~\ref{sec:applications} we give further applications of NIPoPoWs
% beyond the context of a simple client, in particular a \emph{multiblockchain
% wallet} and a \emph{cross-chain ICO}. We also implement our scheme and
% perform measurements to give concrete performance numbers.
% Section~\ref{sec:app-quality} gives lemmas and proofs about the statistical
% properties of chains, which are useful for further results.
% Section~\ref{sec:attack-full} contains
% an attack against these statistical properties, which mandates that a full construction needs to check for them. Section~\ref{sec:security-full} gives a formal
% proof of our security claims through a computational reduction.
% Section~
% \ref{sec:app-succinctness} includes the formal proof that our construction is succinct.
% In Section~\ref{sec:forks}, we illustrate
% gradual deployment paths. One of our techniques allows adoption of our scheme
% without requiring miner consensus. We term this technique a \emph{velvet fork}
% in contrast to the classical \emph{soft} and \emph{hard forks} which require
% approval by a majority of miners. This technique is a novel contribution and may
% be of independent interest for other blockchain protocols.
% We conclude with Section~\ref{sec:variable} which gives an intuition about creating a
% construction for variable difficulty NIPoPoWs by modifying the construction
% presented in this paper.

\import{./}{infix.tex}
% \import{./}{parameterization.tex}
% \import{./}{integration.tex}
% \import{./}{app-quality.tex}
% \import{./}{attack-formal.tex}
% \import{./}{security-formal.tex}
% \import{./}{upgrade.tex}
% \import{./}{variable.tex}
% \import{./}{app-succinctness.tex}
% \import{./}{app-ico.tex}
\import{./}{acknowledgements.tex}
