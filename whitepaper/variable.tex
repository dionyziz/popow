\appsection{Chains of Variable Difficulty}

\label{sec:variable}

In this paper, we have explored constructing PoPoWs always assuming that the
mining target $T$ is constant. In this section, we give the intuition behind a
\emph{variable difficulty} PoPoW construction based on our constant difficulty
construction.

The construction is modified as follows. Assume the chain adjusts its difficulty
in the usual manner as formalized, e.g., in~\cite{backbone2}. In this case, an
\emph{epoch} $e$ is a $0$-subchain of fixed length, say $\ell$, during which the
difficulty remains the same and which is examined in order to perform the
difficulty recalculation. The target has a maximum value in which the difficulty
is the easiest. Suppose this value is $T_0 = \frac{1}{2^\kappa}$ such that all
attempted blocks are valid. We consider the case where the current difficulty is
possible to adjust by factors of $2$; that is, the target of each epoch $e$ is
quantized as $T_e = \frac{T_0}{2^{\mu^*}}$ for some non-negative integer
$\mu^*$, the current \emph{mining level}. In this context, changing the
difficulty is equivalent to saying that the mining game will be played at some
level $\mu^* \geq 0$ and every node considers only chains of that level,
ignoring chains of inferior level. As such, superblock NIPoPoWs can be
constructed as usual, except the used target for NIPoPoW generation and
verification will be the maximum target $T_0$ instead of the current mining
target $T_e$. The only difference is that the honest prover will only provide
chains of level $\mu \geq \mu^*$, as inferior blocks will neither be mined nor
successfully validated by honest participants. The size of variable difficulty
NIPoPoWs is the same as in the constant difficulty case.

The security arguments follow through as before when the comparison of two
proofs $\pi_\mathcal{A}, \pi_B$ happens at some level $\mu_\mathcal{A}, \mu_B
\geq \mu^*$. We must therefore only concern ourselves for the case, during the
comparison, when the adversary presents blocks of level inferior to $\mu^*$
after the LCA block $b$. In the case of an interactive PoPoW, the honest prover
can be informed of the attempt of the adversary to present blocks of level
inferior to $\mu^*$. However, the adversary has committed to the LCA block $b$.
The honest party can subsequently provide a \emph{certificate of hardness} by
presenting the full epoch immediately preceding block $b$, which will consist of
a constant number of blocks, $\ell$. This will allow the honest party to
challenge the adversary's ability to descend to level $\mu_B < \mu^*$ and allow
the verifier to decide that the honest party $B$ should be victorious. This
PoPoW protocol can be made non-interactive by including a certificate of
hardness for every block among the last $m$ blocks of each level presented. This
harms the succinctness by a constant factor of $\ell$ (in Bitcoin,
$\ell = 2016$).

The above sketch gives an intuition of how proofs can be modified to work in a
variable difficulty setting. Formal analysis of its security and succinctness in
various conditions in the model of~\cite{backbone2}, including relaxing the
requirement of quantization $\mu^* \in \mathbb{N}$, will be explored in future
work.
