\section{Introduction}

Bitcoin \cite{bitcoin} is the first decentralized currency. Several
cryptocurrencies have surfaced since, many of which use different consensus
rules. These rules have significant differences with each other. For example,
new coins are often based on proof-of-stake instead of the classical
proof-of-work scheme for mining. This dramatically changes the consensus
mechanism and provides benefits such as efficiency. More minor consensus rules
are also often proposed through altcoins. Many of them provide stronger
anonymity than the first coins through features such as ring signatures or
zero-knowledge proofs. Other changes could involve the block size limits,
smooth emission, or different algorithms for proof-of-stake or proof-of-work.

Such different proposals for consensus rules must necessarily be implemented
either as an altcoin or a hard-fork. This issue was discussed in the sidechains
paper, where the idea of pegged sidechains is proposed: While consensus rules
can be different across blockchains, it is useful to have one unit of value
which can be moved between various blockchains. This allows for experimentation
and potentially powerful ways of deciding consensus rule changes, such as
decentralized voting.

However, in the simplest version of sidechains, the nodes participating in the
consensus protocol must all maintain the different blockchains. This is not
efficient and adoption is not expected. The solution for this is to provide
succinct proofs that the winning chain of a block tree is a particular one.
These proofs are called proofs-of-mining, because they provide a short proof
that mining has happened. In the specific examples of proofs-of-stake and
proofs-of-work, proofs-of-mining are proofs-of-proofs-of-stake and
proofs-of-proofs-of-work. Succinctness requires that these proofs are
significantly shorter than presenting the whole chain. Technically, if the
claimed winning blockchain has a length of $|\chain|$ blocks, we want the succinct
proofs to be bounded by $O(polylg|\chain|)$ in terms of their size in bytes.
These short proofs can then be subsequently included in sidechain blocks where
funds are locked in one blockchain so that they can become available in
another. Note that SPV proofs are still $O(|\chain|)$ but with better constant
factors.

Proofs-of-proofs-of-work were explored in KLS. These proofs were specific to
proofs-of-work and had the disadvantage of requiring interaction between a
prover and a verifier, making them unsuitable for sidechain adoption.

We extend these proofs-of-mining to any kind of mining flavor in a generalize
model which can support proofs-of-proofs-of-work, proofs-of-proofs-of-stake,
hybrids, permissioned and permissionless blockchains and any other consensus
rule choice. We also make the proofs non-interactive, meaning they are now
suitable for sidechain use. We generalize the idea of proof-of-mining into not
just being able to talk about what the current winning chain suffix is, but
allowing provers and verifiers to prove generic predicates. Such predicates can
express simple ideas, such as the inclusion of a transaction in the longest
chain, to complex ideas such as the modification of an account's balance over
time.

We work on the backbone model. We introduce two entities, the Prover and the
Verifier which work in the existing environment. The Prover is part of the
existing miner node which exists in the backbone protocol. The Verifier is a
completely new entity. When the game begins, the Prover and the Verifier are
instantiated with a fixed predicate $q$, which generalized the notion of
proof-of-mining and allows any blockchain property to be proven between the
Prover and the Verifier. Some of these predicates are more suitable for
succinct proofs than others. We focus our attention in what we call \textit{
reliable} predicates; that is, predicates which are monotonic and stable.
These predicates have the exceptional property that all honest miners share
their view of them in a way that is updated in a predictable manner, with a
decision that persists as the blockchain grows. Furthermore, we are interested
in \textit{succinct} predicates, predicates which can be proven using a
succinct protocol, a protocol which requires proofs that are short in terms of
the total blockchain size. Finally, we formalize the notion of
non-interactivity.

We use this model to explore which predicates are reliable and can be proven
succinctly.
