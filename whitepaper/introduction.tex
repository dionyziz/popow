\section{Introduction}

Proof-of-work blockchain clients such as mobile wallets today are based on the
Simplified Payment Verifications (SPV) protocol, which was described in the
original Bitcoin paper~\cite{bitcoin}, and allows them to sychronize with the
network by downloading only block headers and not the entire blockchain with
transactions. However, such initial synchronization still requires receiving all
the block headers. In this work, we study the question of whether better
protocols exist and in particular if downloading fewer block headers is
sufficient to securely synchronize with the rest of the blockchain network. Our
requirement is that the system remains decentralized and that useful facts about
the blockchain (such as a transaction being confirmed) can be
deduced from the downloaded data.

To this end, we put forth a cryptographic security definition for
Non-Interactive Proofs of Proof-of-Work protocols which describes what such a
synchronization protocol must achieve (Section~\ref{sec:model}). We then
construct a protocol which solves the problem and requires sending only a
logarithmic number of blocks from the chain. To aid the presentation, our
protocol is constructed in stages. First, we construct a protocol which can only
synchronize recent blocks, the \emph{suffix proofs} protocol
(Section~\ref{sec:suffix}). Next, we augment the suffix proofs protocol to a
protocol which can synchronize any part of the blockchain that the client may be
interested in, the \emph{infix proofs} protocol (Section~\ref{sec:infix}).
We show that our protocol is secure in our main theorem
(Section~\ref{sec:security}) and that it is also
succinct (Appendix~\ref{sec:succinctness}). Finally, towards a real-world
implementation, we propose concrete parameters for our protocol which are
obtained by experimental simulations (Appendix~\ref{sec:parameters})

%% must verify the entire chain of proofs-of-work, which grows
%% linearly over time. On the contrary, clients based on NIPoPoWs require resources
%% only logarithmic in the length of the blockchain.

%% %
%% Today, Bitcoin and Ethereum remain the two largest proof-of-work
%% cryptocurrencies by market cap. However, the ecosystem has grown diverse, with
%% dozens of viable ``altcoin'' competitors.
%% %
%% Given such an environment,  it becomes increasingly  important to be able to
%% efficiently handle  multiple blockchains by the same client and reliably
%% transfer assets between them.
%
%
%
%We envision our NIPoPoW protocol will form the basis of an
%efficient multi-blockchain client, which could efficiently support payments
%using hundreds of different cryptocurrencies.
%Our protocol improves the performance of client.
%blockchain protocols in general
%are increasingly used as components of larger systems.

% The first objective requires optimizing the ``SPV client'' described in the
% original Bitcoin paper~\cite{bitcoin} which requires processing an amount of
% data growing linearly with the size of the blockchain (namely, the block
% headers).
%
%% Cryptocurrencies such as Bitcoin~\cite{bitcoin}\cite{bitcoinsoftware} and
%% Ethereum~\cite{ethereum} are peer-to-peer networks that maintain a globally
%% consistent transaction ledger, using a consensus protocol based on proof-of-work
%% (PoW) puzzles~\cite{pow,hashcash}. Worker nodes called ``miners'' expend
%% computational work in order to reach agreement on the state of the network.
%% Clients on the network, such as mobile phone apps, must verify these
%% PoWs in order to determine the correct view of the network's state, something necessary
%% to transmit and receive payments correctly.

%% In this work we introduce, analyze and instantiate a new primitive,
%% Non-Interactive Proofs of Proof-of-Work (NIPoPoWs), which can be adapted into
%% existing cryptocurrencies to support more efficient clients. A traditional
%% blockchain client in order to check a certain blockchain property \anote{TODO: too abstract
%
% The second objective has received  significant attention in the context of
% ``cross-chain'' applications, i.e., logical transactions that span multiple
% separate blockchains. Simple cross-chain transactions are feasible today: the
% most well-known is the atomic exchange~\cite{tiernolan,herlihy2018atomic}, e.g.,
% a trade of bitcoin for ether. However, more sophisticated
% applications~
% \cite{interledger,DBLP:journals/corr/DilleyPWPGF16,
%       lerner,drivechains,wood2016polkadot,buchman2016tendermint}
% could be enabled by a more efficient proof process, which would allow the
% blockchain of one cryptocurrency to embed a client of a separate cryptocurrency.
% This concept, initially popularized by a proposal by Back et al.
% ~\cite{sidechains} can be used to avoid a difficult upgrade process: a new
% blockchain with additional features, such as experimental opcodes, can be backed
% by deposits in the original bitcoin currency, obviating the need to transfer
% capital to the new cryptocurrency. As one example of cross-chain interfacing, we
% describe an initial coin offering (ICO)~\cite{ico} which distributes tokens
% issued on one blockchain, but allows paying for them using coins in another
% blockchain.

% These examples illustrate that our solution is a key component for two important
% pillars needed for next-generation blockchains: \emph{interoperability} and
% \emph{scalability}. While we use bitcoin concretely as an example, any
% proof-of-work cryptocurrency can adopt our techniques.

\noindent
\textbf{Our contributions.}
Our main technical contribution is the introduction and instantiation of a new
cryptographic primitive called Non-Interactive Proofs of Proof-of-Work
(NIPoPoWs).

We present a formal model and a provably secure instantiation of NIPoPoWs. We
introduce the first formal security definitions of NIPoPoW protocols and their
security. Our proofs are in the Bitcoin backbone model~\cite{backbone}. Our
construction is the first secure Proof of Proof-of-Work, assuming honest
majority. Furthermore, our solution is non-interactive making it the first
protocol of this kind.

We encode events and other information that is useful to blockchain clients in
the form of \emph{blockchain predicates}. We describe a large class of
predicates that can be proven which we call \emph{infix sensitive}. This enables
proving statements pertaining to the blockchain such as the fact that a
transaction took place (for payment verification), that a smart contract method
ran with certain parameters, or that a payment was made into an account.

We prove the proofs are optimistically succinct meaning that they are
logarithmic in size in honest conditions. In the optimistic model
of no adversarial mining power, succinctness can be achieved for even
\emph{adversarially-generated} proofs.

We provide concrete parameterization and empirical analysis showing the savings
of our approach versus existing clients. Using real data from the Bitcoin and
other networks, we quantify the savings of NIPoPoWs over the previous techniques
of constructing SPV verifiers. For a multi-blockchain client that receives 100
payments per day, we offer a 90\% reduction in bandwidth compared to na\"ive
SPV.

In the Appendix, we propose a novel gradual deployment path for our scheme which
we term a \emph{velvet fork} and which can provide the benefits of our scheme
without requiring miner adoption unlike typical \emph{hard} or \emph{soft
forks}. This technique may be of independent interest for other protocols.

In summary, we make the following contributions:
\begin{enumerate}
  \item We construct the first \emph{provably secure} Proofs of Proof-of-Work
        and give a formal definition of security. We make them
        \emph{non-interactive}.
  \item We model \emph{blockchain predicates} and show which classes of
        predicates can be proven using our protocol.
  \item We create the first decentralized protocol which is secure against all
        adversaries and \emph{succinct} for non-mining adversaries.
  \item We provide \emph{experimental data} which measure the efficiency and
        security of our scheme as well as concrete parameters based on these
        experiments.
  \item We illustrate the practical value of our scheme by discussing further
        applications and presenting a mechanism for backwards-compatible
        deployment into existing blockchains, \emph{velvet forks}.
\end{enumerate}

\noindent
\textbf{Previous work.} The need for succinct clients was first identified by
Nakamoto in his original paper~\cite{bitcoin}. Predicates pertaining to events
occurring in the blockchain have been explored in the setting of
sidechains~\cite{sidechains} without a concrete protocol or definitions. It has
also been implemented for simple classes of predicates such as atomic
swaps~\cite{tiernolan,herlihy2018atomic}, which do not allow full
synchronization. Non-succinct certificates about proof-of-\emph{stake}
blockchains have been proposed in~\cite{gazi2019proof}, but their scheme is not
applicable to proof-of\emph{work}. The idea of superblocks we utilize in our
construction has been first described in the Bitcoin Forum~\cite{highway}. This
was also adopted by~\cite{KLS} to describe their \emph{Proofs of Proof-of-Work}.
Our scheme is comparable to theirs (cf. Section~\ref{sec:suffix}). We improve
upon their work, as their paper does not include a security definition and,
unlike ours, is interactive, which limits its applicability. Additionally, their
scheme is not secure (in fact our attack of Section~\ref{sec:attack} is
sufficient to break their scheme with overwhelming probability) nor succinct
against adversarial provers and does not allow the proof of generic predicates.
