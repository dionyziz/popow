\section{Applications and Evaluation}

\ignore{
\subsection{Comparison with alternatives.}

In Table

\begin{table}
  \caption{Comparison of Sidechains Techniques ($B$ blocks in between transactions)}
  \begin{tabular}{llll}
    {\bf Technique}  & {\bf Cost}       & {\bf Rounds}    & {\bf Trust Model} \\
    Our technique    & $O(m polylog B)$ &              & \\
    PoPoW~\cite{kls} & $O(m polylog B)$ &              & \\
    Plain SPV        &   $O(B)$         & & \\
Drivechains~\cite{sztorc}& $O(1)$       &              & \\
    Federated        &     $O(1)$       &              & Centralized \\
  \end{tabular}
\end{table}

In the case of 
}



\subsection{Multi-blockchain wallets.}
One application of our technique is the potential to make an efficient multi-cryptocurrency client. In this section we estimate the costs

%While every cryptocurrency comes with some form of client software, we do not know of any software that acts as a client for multiple cryptocurrencies.

%Multiple online services provide access to multiple cryptocurrencies. These come with the drawback of relying on a third party.
%Similarly, many multi-asset wallet applications, such as Jaxx\footnote{\url{https://jaxx.io/}} and Coinomi\footnote{\url{https://coinomi.com/}} do not connect to the peer-to-peer networks.

At the current time of writing, a total of 731 cryptocurrencies are listed on \url{https://coinmarketcap.com/}.
This list includes many defunct cryptocurrencies that are no longer functioning at all.
We narrowed our focus to the 80 cryptocurrencies that have their own proof-of-work blockchainsw, (i.e., no proof of stake) and that have a market cap of over USD \$100,000.

In Table~\ref{tbl:currencies} we show aggregate statistics about these 80 crytpocurrencies, grouped according to the their proof-of-work puzzle (e.g., Litecoin uses the scrypt hash function, while Bitcoin uses the SHA-256 function). The ``Size today'' column reflects the curren size of all the proof-of-work headers, while ``Growth rate'' reflects the additional proof-of-work headers generated each day. Here we make the conservative assumption that the size of a block header is exactly 80 bytes, as it is in Bitcoin.\footnote{According to the Ethereum whitepaper~\cite{ethereum}, a header can be approximately 200 bytes.}
 Bitcoin's 10 minute block time is relatively slow; other cryptocurrencies typically use faster block times, e.g. 12 seconds for Ethereum and 2 minutes for Litecoin. Thus while the entire chain of proof-of-work puzzles in Bitcoin today only amounts to 120MB, taken together, the 80 cryptocurrencies comprise nearly 10 GB of proofs-of-work, and generate nearly 10 MB more each day.

 The na\"ive approach to implementing a multi-blockchain client would be to act as an SPV client for each cryptocurrency independently. This approach would entail downloading the header chain for each of the cryptocurrencies, and periodically (e.g., each day) syncing up by fetching any newly generated block headers.

 The NiPoPoW-based client maintains a most recent $k$-stable block hash for each of its supported cryptocurrencies, initially the genesis block for each.
  Each time a payment is received, the client connects to peers on the corresponding network and asks for a NiPoPoW proof relative to the most recently stored block hash. For cryptocurrencies where payments are received very frequently, the NiPoPoW-based client might download nearly every block header, just like an ordinary SPV client; however, for cryptocurrencies used infrequently, the NiPoPoW-based client would be able to skip over many of the blocsk.

 We developed a simulation to evaluate the potential resources savings resulting from the use of a NiPoPoW-based client instead. 
 We model the arrival of payments in each cryptocurrency as a Poisson process, and assume that the market capitalization of a cryptocurrency is a proxy for usage, i.e. in our model most payments received are Bitcoin transactions, with transactions from the remaining 80 constituting a long tail. In all cases, we assume that the parameters are set as described in the previous section, i.e., $m=6$ and $k=24$.
 In Table~\ref{fig:experiment} we show the resulting bandwidth costs from simulating a period of 60 days, with varying rates of payments received.

 For the na\"ive SPV client, the total bandwidth cost is dominated by the need to fetch the entire chain of block headers, which the NiPoPoW client does not need to do. The marginal cost for na\"ive SPV depends on the number of blocks generated each day, as well as the transaction inclusion proofs associated with each payment. The marginal cost of the NiPoPoW payment 

\begin{table}
  \caption{Cost of header chains for all active Proof-of-Work-based cryptocurrencies (collected from \url{Coinwarz.com})}
  \label{tbl:currencies}
\small
  \begin{tabular}{l|l|l|l}
    {\bf Hash} & {\bf \# Instances} & {\bf Size today} & {\bf Growth rate}  \\
    \hline
    Scrypt  & 44  & 4.3 GB  & 5.5 MB / day \  \\
    SHA-256  & 15  & 1.4 GB  & 937.0 kB / day \  \\
    X11  & 5  & 581.1 MB  & 556.3 kB / day \  \\
    Quark  & 3  & 647.9 MB  & 518.4 kB / day \  \\
    CryptoNight  & 2  & 199.0 MB  & 115.2 kB / day \  \\
    EtHash  & 2  & 588.6 MB  & 921.6 kB / day \  \\
    Groestl  & 2  & 300.3 MB  & 184.2 kB / day \  \\
    NeoScrypt  & 2  & 310.6 MB  & 153.6 kB / day \  \\
    Equihash  & 2  & 17.7 MB  & 92.2 kB / day \  \\
    Keccak  & 1  & 161.1 MB  & 115.2 kB / day \  \\
    X13  & 1  & 30.0 MB  & 57.6 kB / day \  \\
    Lyra2REv2  & 1  & 57.4 MB  & 46.1 kB / day \  \\
    \hline
    Total  & 80   &  8.5 GB  & 9.2 MB  / day  \\
  \end{tabular}
\end{table}



\begin{table}
  \caption{Simulated bandwidth of multi-blockchain clients after two months (Averaged over 10 trials each)}
  \small
  \begin{tabular}{l|ll|ll|l}
               & \multicolumn{2}{c|}{Naive SPV} & \multicolumn{2}{c|}{NiPoPoW} \\
    Tx/day & Total & (Per day) & Total & (Per day) & Improvement \\
    \hline
    100        & 5.5 GB          & (5.5 MB) & 25.6 MB & (416.5 kB) & (0.08) \\
    500        & 5.5 GB          & (5.7 MB) & 35.7 MB & (578.6 kB) & (0.10) \\
    1000       & 5.5 GB          & (6.0 MB) & 48.8 MB & (792.9 kB) & (0.13) \\
    \end{tabular}
  \end{table}

\subsection{More Applications}

\paragraph{Catena.}

In Catena,~\cite{} propose development of.

\anote{TODO: Copy cost estimates and usage experiments from Catena paper.}

Develop a header relay network as a way of releasing.
If relay header network instead provided proofs based on NiPoPoWs, 

\paragraph{Sidechains.}


