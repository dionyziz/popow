\section{Evaluation \& Applications}
\label{sec:applications}
In this section we evaluate the cost of NIPoPoWs when used in realistic blockchain applications.
First we simulated the resources savings resulting from the use of a NIPoPoW-based
client compared to ordinary SPV. We model the arrival of payments in each cryptocurrency as a Poisson process
and assume that the market cap of a cryptocurrency is a proxy for usage. Currently,
a total of 731 cryptocurrencies are listed on coin market
directories\footnote{\url{https://coinmarketcap.com/}}. We narrow our focus to
the 80 cryptocurrencies that have their own PoW blockchains with
a market cap of over USD \$100,000.

In Table~\ref{tbl.currencies} we show aggregate statistics about these 80
cryptocurrencies, grouped according to the their PoW puzzle. While the entire
chain in Bitcoin only amounts to 40 MB, taken together, the 80 cryptocurrencies
comprise 10 GB of proofs-of-work, and generate 10 MB more each day. In
Table~\ref{tbl.experiment} we show the resulting bandwidth costs from simulating
a period of 60 days with $m=24, k=6$, with varying rates of payments received.
%
For the na\"ive SPV client, the total bandwidth cost is dominated by fetching
the entire chain of headers, which the NIPoPoW client avoids. The marginal
cost for na\"ive SPV depends on the number of blocks generated per day, as well
as the transaction inclusion proofs associated with each payment. The
NIPoPoW-based client provides the most savings when the number of transactions
per day is smallest, reducing the necessary bandwidth per day (excluding the
initial sync up) by 90\%.

\begin{table}
  \caption{Cost of header chains for all active PoW-based cryptocoins
           (collected from \url{coinwarz.com})}
  \vspace{-1em}
  \label{tbl.currencies}
  \small
  \centering
  \begin{tabular}{l@{}|l|l|l@{}}
    {\bf Hash} & {\bf Coins} & {\bf Size today} & {\bf Growth rate}  \\
    \hline
    Scrypt  & 44  & 4.3 GB  & 5.5 MB / day \  \\
    SHA-256  & 15  & 1.4 GB  & 937.0 kB / day \  \\
    X11  & 5  & 581.1 MB  & 556.3 kB / day \  \\
    Quark  & 3  & 647.9 MB  & 518.4 kB / day \  \\
    CryptoNight  & 2  & 199.0 MB  & 115.2 kB / day \  \\
    EtHash  & 2  & 588.6 MB  & 921.6 kB / day \  \\
    Groestl  & 2  & 300.3 MB  & 184.2 kB / day \  \\
    NeoScrypt  & 2  & 310.6 MB  & 153.6 kB / day \  \\
    Others  & 5  & 266.2 MB  & 311.1 kB / day \  \\
    % "Others" above is the sum of these:
    % Equihash  & 2  & 17.7 MB  & 92.2 kB / day \  \\
    % Keccak  & 1  & 161.1 MB  & 115.2 kB / day \  \\
    % X13  & 1  & 30.0 MB  & 57.6 kB / day \  \\
    % Lyra2REv2  & 1  & 57.4 MB  & 46.1 kB / day \  \\
    \hline
    Total  & 80   &  8.5 GB  & 9.2 MB  / day  \\
  \end{tabular}
\vspace{-1em}
\end{table}

\begin{table}
  \caption{Simulated bandwidth of multi-blockchain clients after two months (Averaged over 10 trials each)}
  \vspace{-1em}
  \label{tbl.experiment}
  \small
  \centering
  \begin{tabular}
    {
      l@{\hspace{1pt}}|
      l@{\hspace{1pt}}l@{\hspace{1pt}}|
      l@{\hspace{1pt}}l@{\hspace{1pt}}|
      l@{\hspace{1pt}}}

      \multicolumn{1}{@{}l|}{\bf tx/} & \multicolumn{2}{c|}{\bf Naive SPV} & \multicolumn{2}{c|}{\bf NIPoPoW} \\
      {\textbf{day}} & {\bf Total} & {\bf (Daily)} & {\bf Total} & {\bf (Daily)} & {\bf Savings} \\
    \hline
    100   &  5.5 GB & (5.5 MB)   & 31.7 MB & (507 kB)   & 99\% (91\%) \\
    500   &  5.5 GB & (5.7 MB)   & 68.2 MB & (1.1 MB)     & 99\% (81\%) \\
    1000  &  5.5 GB & (6.0 MB)   & 99.1 MB & (1.6 MB)     & 98\% (73\%) \\
    3000  &  5.6 GB & (7.0 MB)   & 192 MB& (3.1 MB)     & 97\% (56\%) \\
    \end{tabular}
  \vspace{-2em}
  \end{table}

\noindent
\textbf{Multi-blockchain wallets.}\label{sec:multichain}
An application of our technique is an efficient multi-cryptocoin client.
Consider a merchant who wishes to accept payments in any cryptocoin, not just
the popular ones. The na\"ive approach would be to install an SPV client for
each known coin. This approach would entail downloading the header chain for
each coin, and periodically syncing up by fetching any newly generated block
headers. An alternative would be to use an online service supporting multiple
currencies, but this introduces reliance on a third party (e.g. Jaxx and Coinomi
rely on third party networks).

A NIPoPoW-based client would not download the entire header chain, but would
instead only receive NIPoPoW proofs each time a payment is received. When a peer
informs the client about a payment, they include a block index $\ell$ and
NIPoPoW proof of transaction inclusion. The peer must then query \emph{all} of
their connected peers, requesting any better proof for the same predicate. After
waiting a short time period for a response, the client runs the
\texttt{verify-infix} routine on all received proofs, and accepts the
transaction if the output is \emph{true}. Although initially such proofs must be
relative to genesis, the client may store the most recently-known ($k$-stable)
blockhash for each coin such that future payments can include NIPoPoW proofs
relative to that. Thus for popular cryptocoins, the NIPoPoW-based client
downloads nearly every block header, like an ordinary SPV client; but for
coins used infrequently, the NIPoPoW-based client can skip over most blocks.

\noindent
\textbf{Cross-chain ICOs.}
As an example use-case of our construction, we present the case of an
ICO in which tokens are distributed in one blockchain,
but funds are raised in another. It works as follows:
There are two designated blockchains, the \emph{source} and
the \emph{destination blockchain}. The source is the blockchain
where the fund-raising will take place, while the destination is the
blockchain where the newly issued tokens will be distributed and subsequently
exchanged. The destination blockchain must be smart-contract-enabled in order to
allow for the distribution of ERC-20-style~\cite{erc20} tokens.
In addition, the smart contracts on the destination blockchain must allow for
programming the verification of a NIPoPoW proof by including, for example, the
appropriate hash functions. The source blockchain must be NIPoPoW-enabled. This
setup allows the creation of NIPoPoWs \emph{about} the source blockchain which
will be included in the destination blockchain. For example, a source blockchain
can be Litecoin and a destination blockchain Ethereum.

In order to run the ICO, the fund-raising entity first creates a designated
account in the source blockchain in which funds will be deposited. It then
creates the ERC-20-style smart contract in the destination blockchain. When
someone wishes to participate in the ICO, they transfer funds into the
designated account on the source blockchain. Once they have made the transfer
and it becomes confirmed, the payer generates a NIPoPoW about the transaction
paying into the designated account. This NIPoPoW is then sent as a parameter to
a method call on the ICO smart contract on the destination blockchain. The
method call stores the proof and waits for a certain period of time for possible
contestations, which can be accepted and compared using the $\leq_m$ mechanism
previously described. If no contesting proof is presented within the
contestation period, the prover receives their respective ICO tokens on the
target blockchain. In order for only the rightful owner to be able to receive
the tokens, they are required to sign the destination address on the destination
blockchain using the private key corresponding to their source account used to
make the payment within the source blockchain.

%\subsection{Compare proofs implementation details}
