\begin{proof}
    We will prove security by reductio ad absurdum. Assume a typical execution.
    Let $\overline{\Pi}_A, \overline{\Pi}_B$ be an adversarial and an honest
    proof respectively, and assume the security property is violated. That is,
    the honest verifier is convinced that $Q(\chain)$ has a value different
    from all honest full nodes.

    Let $r_3$ be the round during which the honest proof $\overline{\Pi}_B$ was
    constructed; that is, the current round. $\overline{\Pi}_B$ will be a proof
    constructed based on an underlying blockchain $\chain_B$.

    Let $b$ be the LCA of $\overline{\Pi}_A$ and $\overline{\Pi}_B$ and assume
    that $A$ is the adversary. Let $\mu$ be the highest level at which $A$
    is able to present $m$ blocks of level $\mu$ after $b$. We will show that
    if such a level exists, then the winning proof will necessarily be
    $\overline{\Pi}_B$.

    Indeed, note that, because $A$ is winning, therefore $\mu$ is the highest
    level at which any of the parties was able to present at least $m$ blocks
    of that level after $b$. This means that at every higher level $\mu + i$
    for $i > 0$, it was true that $b$ was within the last $m$ blocks of that
    level.  Therefore, $B$, due to the way the proofs are constructed, would
    have included all the blocks following $b$ of that level. Hence, all the
    blocks of the honest chain following $b$ of level $\mu$ have been included
    in the proof $\overline{\Pi}_B$.

    Let $\chain'_A$ be the adversarial $\mu$-level superchain after block $b$
    and $\chain'_B$ be the honest party's $\mu$-level superchain after block
    $b$.  These two chains will be disjoint since $b$ is defined as the LCA.

    We will now show that $\chain'_A[k + 1:]$ contains no honestly mined
    blocks. By contradiction, assume that $\chain'_A[i]$ for some $i > k + 1$
    was honestly generated. But this means that an honest party had adopted the
    chain $\chain'_A[i - 1]$ at some round $r_2 \leq r_3$. Because of the way the
    honest parties adopt chains, this means that the superchain $\chain'_A[:i -
    1]$ has an underlying properly constructed $0$-level chain $\chain_A$ which
    starts at genesis and includes all the blocks of the superchain
    $\chain'_A[:i - 1]$. Let $j$ be the index of block $b$ within $\chain_A$.
    As $\chain_A$ contains at least all the blocks of the superchain
    $\chain'_A$, observe that $|\chain_A[j + 1:]| \geq i - 1 > k$.
    Therefore $\chain_A[:-k] \not\preccurlyeq \chain_B$. But, as $\chain_A$ was
    adopted by an honest party at round $r_2$ which is prior to round $r_3$
    during which $\chain_B$ was adopted by an honest party, this contradicts
    the Common Prefix property of blockchains. It follows that the $m - k$ last
    blocks of the adversarial proof have been adversarially mined.

    Let $b'$ be the most recent honest block preceeding $b$, which could be the
    genesis block and let $r_1$ be the round during which $b'$ was generated.
    We will consider the set of consecutive rounds $S$ starting at round $r_1$
    and ending at round $r_3$. We are considering a superchain of level
    $\mu$ which has at least $m$ blocks. From the assumption that the
    execution is typical, it follows that $Z^\mu(S)$ and $X^\mu(S)$ are
    distributed close to their respective means. The probability that the
    adversary is able to produce $m$ $\mu$-level blocks after block $b'$ faster
    than the honest parties are able to produce $m$ $\mu$-level blocks after
    block $b'$ is negligible.

    Therefore, if there exists such a level $\mu$, then the honest party proof
    will necessarily win. If there is no such level $\mu$, then the comparison
    happens at the $0$-level, and hence reduces to the security of a full
    verifier.
\end{proof}
