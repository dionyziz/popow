\section{Constructing blockchain proofs}

We introduce a helper algorithm, ConstructInnerChain, shown in
Algorithm~\ref{alg.nipopow_construct_innerchain}. This algorithm returns the innerchain
of level $i$ extracted from the blockchain $\chain$. If boundary is provided,
it only returns the blocks more recent than the boundary block supplied.

\import{./}{algorithms/alg.nipopow-prover.tex}

The NiPoPoW proof construction is shown in Algorithm~\ref{alg.nipopow_construct_proof}.
This produces a non-interactive PoPoW in parameter $m$ which consists of a
number of blocks for every level $i$. The number of blocks per level is
approximately $2m$.

\begin{figure}[h]
    \caption{The hierarchical blockchain. Existing blocks are shown in level 1.
    Higher levels have achieved a lower target (higher difficulty) during mining.}
    \centering
    \includegraphics[width=0.5\textwidth,keepaspectratio]{figures/hierarchical-ledger.png}
    \label{fig:hierarchy}
\end{figure}

\begin{figure}[h]
    \caption{The first of a series of interactive proofs-of-proofs-of-work for
    $m = k = 3$. This proof is the only one that needs to be sent in case it
    goes unchallenged.}
    \centering
    \includegraphics[width=0.5\textwidth,keepaspectratio]{figures/interactive-popow.png}
\end{figure}

\begin{figure}[h]
    \caption{A non-interactive proof-of-work for $m = k = 3$. Any challenges
    can be answered by the verifier directly by constructing a proof from the
    data in this proof, without interaction with the prover.}
    \centering
    \includegraphics[width=0.5\textwidth,keepaspectratio]{figures/non-interactive-popow.png}
\end{figure}
