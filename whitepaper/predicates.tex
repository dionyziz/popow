\subsection{Provable chain predicates}

Our aim is to prove statements about the blockchain, such as ``The transaction
$t$ is included in the current blockchain.'' We consider a general class of
predicates that take on values \emph{true} or \emph{false}.  Since a
Bitcoin-like blockchain can experience delays and intermittent forks, not all
honest parties will be in exact agreement about the entire chain. However, when
all honest parties are in agreement about the truth value of the predicate, we
will soon require in our security definition that the verifier also arrives at
the same truth value.

To aid the construction of our proofs, we focus on predicates that are
\emph{monotonic}; they start with the value \emph{false} and, as the blockchain
grows, can change their value to \emph{true} but not back.

\begin{definition}{(Monotonicity)}
    A chain predicate $Q(\chain)$ is $\textit{monotonic}$ if for all chains
    $\chain$ and for all blocks $B$ we have that
    $Q(\chain) \Rightarrow Q(\chain B)$.
\end{definition}

Additionally, we require that our predicates only depend on the \emph{stable}
portion of the blockchain, blocks that are buried under $k$ subsequent blocks.
This ensures that the value of the predicate will not change due to a blockchain
reorganization.

\begin{definition}{(Stability)}
    Parameterized by $k \in \mathbb{N}$, a chain predicate $Q$ is
    $k$-\emph{stable} if its value only depends on the prefix $\chain[:-k]$.
\end{definition}
