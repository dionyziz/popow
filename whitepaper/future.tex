\section{Future work}

Our security proof was constructed in the Backbone \cite{backbone} setting in
which the block difficulty is treated as a constant.  It is easy to extend our
construction to the variable difficulty setting.  The only difference is that
chains (and proofs)  would have to be compared according to the total difficulty
they possess, i.e.,  blocks at superlevel $\mu$ will by $\mu$ times more
difficult than $0$-level blocks. We conjecture that this construction can be
proven secure using a similar argumentation as the one presented. The model in
\cite{backbone2} can be used as a basis and   we leave the full analysis  for
future work.

We worked with a construction that allows a certain class of chain predicates to
be proved: In particular, stable predicates which are functions only of the
suffix of the chain or depend on a small segment within the chain. However,
slightly more general predicates can also be proved succinctly. We leave the
complete characeterization of succinctly provable predicates for future work.

We have constructed Proofs-of-Proofs-of-Work. An open question remains whether
the same problem of Proof-of-Proofs is solvable in Proof-of-Stake settings, a
Proof-of-Proof-of-Stake (PoPoS) protocol. It is unknown whether this problem can
be solved interactively or non-interactively.

Finally, while our construction hints at a direction of sidechains, no formalism
exists that allows us to reason about whether NIPoPoW protocols are appropriate
as sidechain solutions. A proper security definition for the desirable
properties of sidechains would allow this evaluation to take place.
