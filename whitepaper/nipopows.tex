\documentclass[11pt]{llncs}
\usepackage{preamble}

\begin{document}
\import{./}{title.tex}
\thispagestyle{plain}
\import{./}{abstract.tex}


\import{./}{oracle.tex}

\section{The Nakamoto protocol}

``Honest parties always work to extend the longest valid blockchain.''

Validity is defined inductively as follows:
- A puzzle solution with no outgoing edges is valid.
- A puzzle solution with a single outgoing edge to a valid puzzle solution is valid.

Properties (all of which hold when $f/N < 50\%$):
    Chain Quality:  the longest chain consists of puzzle solutions generated by honest nodes
    Prefix consensus: the longest valid chain includes a prefix that does not change.
        Since the adversary cannot add more puzzle solutions in a round than the honest chain, then the honest parties’ chain grows faster.

\section{Stateless Verification}

First, an arbitrary node from the graph (public or private) chain is chosen.
The verifier’s goal is to determine whether the node is ``stable,'' i.e. it is included in the longest chain.

Honest nodes and the adversary take turns presenting proofs to the verifier. The verifier does not use the Read() interface of the oracle, but simply accepts proofs sent to it by the oracle.

The verifier cannot tell which proofs are sent from honest parties or from the adversary!

Goals:
- (Common prefix) If the verifier accepts a transaction, then it will remain in the longest valid chain.
- (Liveness) If a node is included the chain, then after a short time GenerateProof() will produce a proof that the verifier accepts, regardless of proofs submitted by the adversary.

Bitcoin SPV Proofs:
	Simply pass the entire chain.
	In practice, the blockchain contains labels called “merkle roots”, and validity is based on whether merkle root corresponds to a set of valid transactions. Thus an SPV is substantially smaller than the entire blockchain including all the transactions. However, it grows linearly as the proof-of-work chain grows.

\section{Interactive Stateless Verification}

The Nakamoto protocol is modified as such:
- Every node still works to extend the longest valid chain.

A ``score'' is associated to, based on the number of leading 0-bits in the
``score'' label. Thus $X = score(n)$ is distributed according to a geometric distribution (with parameter 0.5).

Validity is modified as such:
- Each node contains several links. One is to the most recent puzzle solution, similar to the original protocol.
- Other links are to the most recent puzzle solution with $score(n) \geq
1$,  with $score(n) \geq 2$, and so on.

Present a proof-of-proof-of-work $(X, \pi)$. $\pi^{\rceil k}$

Disputing a false proof:
    If the transaction in question is *not* in the longest valid chain, but the adversary presents a proof.

Why is interaction needed?

Links along the layer.

\section{Compact SPV Proofs}

Similar to Kiayias, we utilize backlinks, in effectively the same structure.

Definitions:
$(\alpha, k)$ - Segment proof for $[start, end]$

Properties:
with high probability, at least puzzle solutions that contain ``start'' as a descendant

Efficiency:
	honest parties are able to produce a proof that contains fewer than $x$ blocks

Prefix proof.  for [end]
    The idea is to have successively larger proofs.

Let $A$ be a valid puzzle chain.

$A_{f}^{\mu}$, where $f \in A$, consists of the suffix of $A$ after
$f$, such that $n \in A$, $score(n) > \mu$, and $f \subset_A n$.

$\Pi_A[f]$  is defined by Alg 1.

$\Pi_{A}$ is a pruned version of A, defined by Alg 1.
$\Pi_{A}$ satisfies the following property:

\begin{equation*}
    \exists \mu \texttt{ such that } \Pi_{A}{f}^{\mu} = A_f^{\mu}
    \texttt{ and }
    |A_f^{\mu}| > k
    \texttt{ or } \mu
\end{equation*}

\section{Constructing non-interactive proofs}

A non-interactive proof-of-proof of work is a game between a Prover and a
Verifier parameterized by predicate $q$ which the prover tries to convience the
verifier for. The predicate $q$ is a function of a chain $\chain$, which the
prover claims to be the longest chain.

\import{./}{algorithms/alg.backbone.tex}
\import{./}{algorithms/alg.verifier-framework.tex}
\import{./}{algorithms/alg.verifier-full.tex}
\import{./}{algorithms/alg.verifier-lite.tex}

At the beginning of the game, two Provers generate proofs $\pi_A$, $\pi_B$
claiming potentially different truth values for the predicate $q$ based on
their claimed local longest chains. The Verifier receives these proofs and
accepts one of the two proofs, determining the truth value of the predicate.

We say that a predicate $q$ is $\textit{reliable}$ if it has the following
desirable properties:

\begin{itemize}
    \item The predicate is $\textit{monotonous}$ for honest provers: For a
        sequence of chains $\chain _1, \chain _2, \ldots$ of an honest prover, we
        have the there is some $i$ such that the predicate $q(\chain _j)$ is
        false for all j before that i; and subsequently, the predicate
        $q(\chain _j)$ becomes true for $j \geq i$ and remains true for ever.
        It is legal for $i$ to be at infinity, making the predicate false
        throughout.
    \item The predicate has $\textit{liveness}$: Once the environment puts
        adequate data in the blockchain payload for the predicate to become
        true, it will become true for all honest players.
    \item The predicate has $\textit{common prefix}$: Once it becomes true for
        one honest player, the rest will eventually follow.
\end{itemize}

\import{./}{algorithms/alg.nipopow-innerchain.tex}

We introduce a helper algorithm, ConstructInnerChain, shown in
Algorithm~\ref{alg.nipopow_construct_innerchain}. This algorithm returns the innerchain
of level $i$ extracted from the blockchain $\chain$. If boundary is provided,
it only returns the blocks more recent than the boundary block supplied.

\import{./}{algorithms/alg.nipopow-prover.tex}

The NiPoPoW proof construction is shown in Algorithm~\ref{alg.nipopow_construct_proof}.
This produces a non-interactive PoPoW in parameter $m$ which consists of a
number of blocks for every level $i$. The number of blocks per level is
approximately $2m$.

\begin{figure}[h]
    \caption{The hierarchical blockchain. Existing blocks are shown in level 1.
    Higher levels have achieved a lower target (higher difficulty) during mining.}
    \centering
    \includegraphics[width=\textwidth,keepaspectratio]{figures/hierarchical-ledger.png}
    \label{fig:hierarchy}
\end{figure}

\begin{figure}[h]
    \caption{The first of a series of interactive proofs-of-proofs-of-work for
    $m = k = 3$. This proof is the only one that needs to be sent in case it
    goes unchallenged.}
    \centering
    \includegraphics[width=\textwidth,keepaspectratio]{figures/interactive-popow.png}
\end{figure}

\begin{figure}[h]
    \caption{A non-interactive proof-of-work for $m = k = 3$. Any challenges
    can be answered by the verifier directly by constructing a proof from the
    data in this proof, without interaction with the prover.}
    \centering
    \includegraphics[width=\textwidth,keepaspectratio]{figures/non-interactive-popow.png}
\end{figure}

\import{./}{references.tex}
\end{document}
